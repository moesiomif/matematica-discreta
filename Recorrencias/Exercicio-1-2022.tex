\documentclass[11pt,a4paper]{article}
\usepackage{moesio}
\usepackage{caption}
%==========================================================================================
\newcommand{\nomet}{\bf Msc. Moésio}
\newcommand{\nome}{\bf Moésio M. de Sales}
\newcommand{\nomer}{Recorrências}
\newcommand{\titu}{Atividade 1 - Recorrências}
\newcommand{\disc}{Matemática Discreta}
\newcommand{\curso}{Sistemas de Informação}
\newcommand{\inst}{IFCE}
%==========================================================================================
\begin{document}
\Large
\begin{center} \titu\\ \disc\\ \curso\\  \nome\footnote{moesio@ifce.edu.br} \end{center}
%==========================================================================================
\framebox[8.1\width]{Alun@:\hfill}\fbox{\today}\\
%\hrule\ 
%==========================================================================================
\fbox{\begin{minipage}{34.0em}
 Respostas sem justificativas não serão consideradas na correção. {\bf 1 Ponto}
\end{minipage}
}

\bexer
\item Se $x_{n+1}=2x_n$ e $x_1=3$, determine $x_n$.
\item Se $x_{n+1}=x_n+3$ e $x_1=2$, determine $x_n$
\item Se $x_{n+1}=5x_n+3$ e $x_1=1$, determine $x_n$

\item Resolva a equação $x_{n+1}=(n+1)x_n+n$, $x_1=1$.
\item Resolva a equação $(n+1)x_{n+1}+nx_n = 2n-3$, $x_1=1$.
\item Resolva a equação $x_{n+1}-nx_n=(n+1)!$, $x_1=1$.
\item 
		Para sequências definidas por $x_{n+2}=2x_{n+1}+x_n$, $x_0=x_1=1$, determine $x_{50}$.

\item 
		Para sequências definidas por $x_{n+2}=7x_{n+1}-10x_n$, $x_0=x_1=1$, determine $x_{20}$.

\item 
%Para sequências definidas por $x_{n+2}=2x_{n+1}+3x_n$, $x_0=x_1=1$, determine $x_{10}$.
%Para sequências definidas por $x_{n+2}=6x_{n+1}-8x_n$, $x_0=x_1=1$, determine $x_{10}$.
Para sequências definidas por $x_{n+2}=6x_{n+1}-5x_n$, $x_0=x_1=1$, determine $x_{10}$.

\item Quantas são as sequências com $n$ letras, cada uma igual a $a$, $b$ ou $c$, de modo que não há
  duas   letras $a$  seguidas?
%$x_{n+2}=2x_{n+1}+2x_n$

%elon pag 72
\item Quantas são as sequências de $n$ termos, todos pertencentes a $\{0,1\}$, que possuem  um
  número ímpar de termos iguais a $0$?

%O número de sequências é a soma do número de sequências começadas por $1$ com o número de
%sequências começadas por $0$. isto é. $x_{n+1}=x_n+(2^n-x_n)$

%execio Recorrênicas Segunda Ordem
\item  Resolver a recorrência $x_n = 6x_{n-1} - 8x_{n-2} + 2$ , com $x_1 = x_2 = 1$.
%x_n=\frac{2^n}{4}-\frac{4^n}{24}+\frac{2}{3}$

\item Resolva a equação a seguir.
\[x_{n+2}-5x_{n+1}+6x_n=n+3^n ,\ x_0=1,\ x_1=2\].

\item %elon Q2-a pag 81
Resolva a equação
\[x_{n+2}+5x_{n+1}+6x_n=0,\ x_0=3;\ \ x_1=-6\]
\item{(MMS-2018-Adaptado)}  Durante a guerra de judeus e romanos, Josephus estava entre rebeldes judeus encurralados
	  em uma   caverna pelos romanos. Preferindo o suícidio a captura, os rebeldes decidiram formar 
	  um círculo e, contando ao longo deste, suicida-se uma pessoa sim, uma não, até não sobrar
	  ninguém. 
	  \begin{enumerate}
		\item Se na caverna estavam $11$ rebeldes. Determine qual posição Josephus deveria escolher para sair ileso desse círculo malígno.
		\item Se na caverna estavam $n$ rebeldes. Determine qual posição Josephus deveria escolher para sair ileso desse círculo malígno.
	  \end{enumerate}
	  % 8 posição
	  %$x_n =2^k$ onde $k$ é o maior inteiro tal que  $2^k \leq n$ ou seja, $x_n = 2^{\lfloor \frac{\log n}{\log 2}\rfloor}$


\item  Caminhando pelos segmentos unitários da figura abaixo, determine quantas são as maneiras de
  ir de $A$ até $B$   sem passar duas vezes pelo mesmo ponto.\label{q4-rec}
  \begin{minipage}[h]{0.04\linewidth}
	\centering
	\includegraphics[scale=1]{fig1-q4-rec2}
  \end{minipage}


%texto
\item O DNA de um aluno de discreta é formado por sequências de cinco proteínas chamadas $0,\ 1,\
  c,\ x$, e $y$. Nas sequências, nunca aparecem $cx$, $cy$, $yx$ e $yy$. Todas as outras
  possibilidades são permitidas. 
  \begin{enumerate}
	%\item Crie um programa em C\texttt{++} que leia $n$ e liste todas sequências de $n$ termos conforme o enunciado.
	\item Quantas são as possíveis sequências de DNA de um aluno de discreta com
  $n$ proteínas?

  {\bf Sugestão:} Sejam $a_n$ a quantidade de sequências de $n$ proteínas que não terminam com $c$ ou
  $y$ e $b_n$ a quantidade de sequências que terminam com $c$ ou $y$. Queremos $x_n=a_n + b_n$ (Por quê?).
				  \[\begin{cases}
							a_0=1,\ b_0=0 \\
							a_n=3a_{n-1}+2b_{n-1}\\
							b_n= 2a_{n-1}+b_{n-1}
					\end{cases}
				  \]
				Note que, $a_1=3$ e $b_1=2$(Por quê?). Isolando as variáveis\ldots
%				\[
%				  \begin{cases}
%				 			a_0=1,\ b_0=0 \\
%							b_{n-1}=\frac{a_n -3a_{n-1}}{2}\\
%							a_{n-1}=\frac{b_n - b_{n-1}}{2}
%				  \end{cases}
%				\]
%		Como vale para todo $n$, podemos substituir:
%		
%				\[
%				  \begin{cases}
%				 			a_0=1,\ b_0=0 \\
%							b_{n-1}=\frac{ \frac{b_{n+1} - b_{n}}{2} -3\frac{b_n - b_{n-1}}{2}}
%{2}\\
%a_{n-1}=\frac{\frac{a_{n+1}-3a_n}{2} - \frac{a_n-3a_{n-1}}{2}}{2}
%				  \end{cases}
%				  \Leftrightarrow 
%				  \begin{cases}
%					a_0=1,\ b_0=0\\
%					b_{n+1}=4b_{n}+b_{n-1}\\
%					a_{n+1}=4a_{n}+a_{n-1}\\
%				  \end{cases}
%				\]
%				Complete a solução achando as fórmulas fechadas para $a_n$ e $b_n$.
%
	\item Quantas são as possíveis sequências de DNA de um aluno de discreta com $10$ proteínas?
  \end{enumerate}
  
  %Uma sonda espacial descobriu que o material orgânico em Marte tem DNA composto de cinco símbolos, denotado por (a, b, c, d, e), em vez dos quatro componentes do DNA terráqueo. Os quatro pares cd, ce, ed e ee nunca ocorrem consecutivamente em uma string de DNA marciano, mas qualquer string sem pares proibidos é possível. (Assim, bbcda é proibida, mas bbdca é OK.) Como casar com seqüências de DNA de Marte de comprimento n são possíveis? (Quando n = 2 a resposta é 21, porque as extremidades esquerda e direita de uma string são distinguíveis.)
%Solução pag.559 

%leno vol.2 mat esn. med.
\item O salário de Carmelino no mês $n$  é $S_n =10+3n$. Sua renda mensal é formada pelo salário e
pelos juros de suas de suas aplicações financeiras. Ele poupa anualmente $1/p$ de sua renda e
investe sua poupança a juros mensais de taxa $i$. Determine a renda de Carmelino no mês $n$.

{{\bf Sugestão:} $x_n=S_n +iy_{n-1}$, $y_n=y_{n-1}+\frac{1}{p}x_n$, onde $y_n$ e o montante da
poupança no final do mês $n$. Tire o valor de $y$ na primeira equação e substitua na segunda.}


\eexer

\end{document}
