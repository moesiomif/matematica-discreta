%leno vol.2 mat esn. med.
\item O salário de Carmelino no mês $n$  é $S_n =10+3n$. Sua renda mensal é formada pelo salário e
pelos juros de suas de suas aplicações financeiras. Ele poupa anualmente $1/p$ de sua renda e
investe sua poupança a juros mensais de taxa $i$. Determine a renda de Carmelino no mês $n$.

{{\bf Sugestão:} $x_n=S_n +iy_{n-1}$, $y_n=y_{n-1}+\frac{1}{p}x_n$, onde $y_n$ e o montante da
poupança no final do mês $n$. Tire o valor de $y$ na primeira equação e substitua na segunda.}
